\documentclass[11pt]{article}
\setlength{\oddsidemargin}{0.1 in}
\setlength{\evensidemargin}{-0.1 in}
\setlength{\topmargin}{-0.75 in}
\setlength{\textwidth}{6.5 in}
\setlength{\textheight}{9 in}
\setlength{\headsep}{0.75 in}
\setlength{\parindent}{0 in}
\setlength{\parskip}{0.2 in}

\usepackage{amsmath,amsfonts,graphicx,float, amssymb}
\usepackage{enumitem}
\usepackage[english]{babel}
\usepackage[square,numbers]{natbib}
\bibliographystyle{plainnat}
\usepackage{tikz}

\begin{document}
\section{Notation}
We make use of the following notation in this report.
\begin{itemize}
\item $r,p,o,m$ in subscript: Robot, particle, odometry and map frames, respectively
\item $R_{r,o}$: Rotation matrix expressing the orientation of the robot frame $r$ in the odometry frame $o$.
\item $R_{p,m}$: Rotation matrix expressing the orientation of the particle frame $p$ in the map frame $m$.
\end{itemize}

\section{Approach}
The following sections outline our approach in implementing the particle filter.

\subsection{Motion Model}
\begin{enumerate}
\item Given subsequent odometry readings $\left(x,y,\theta\right)_{1}$ and $\left(x,y,\theta\right)_{2}$ at $t_{1}, t_{2}$, we first compute the translation in the odometry frame:
\begin{align*}
\begin{bmatrix}
\Delta x \\
\Delta y
\end{bmatrix}_{o} = 
\begin{bmatrix}
x_{2} - x_{1} \\
y_{2} - y_{1}
\end{bmatrix}
\end{align*} 

\item We then rotate the displacements to the robot frame (computed using odometry values):
\begin{align*}
\begin{bmatrix}
\Delta x \\
\Delta y
\end{bmatrix}_{r} = 
R_{r}^{-1}
\begin{bmatrix}
\Delta x \\
\Delta y
\end{bmatrix}_{o} 
\end{align*}

\item We add some noise to get corresponding displacements in the particle frame.
\begin{align*}
\begin{bmatrix}
\Delta x \\
\Delta y
\end{bmatrix}_{p} = 
\begin{bmatrix}
\Delta x \\
\Delta y
\end{bmatrix}_{r} +
\begin{bmatrix}
\mathcal{N}\left( 0, \sigma_{motion} \right) \\
\mathcal{N}\left( 0, \sigma_{motion} \right)
\end{bmatrix}_{}
\end{align*}
Similarly, the difference in $\theta$ is computed according to:
\begin{align*}
\Delta \theta = \theta_{2} - \theta_{1} + \mathcal{N}\left( 0, \sigma_{angle} \right)
\end{align*}

\item We rotate the displacement from the particle frame to the map frame:
\begin{align*}
\begin{bmatrix}
\Delta x \\
\Delta y
\end{bmatrix}_{m} = 
R_{p} * \begin{bmatrix}
\Delta x \\
\Delta y
\end{bmatrix}_{p}
\end{align*}

\item Finally, we update the particle's pose in the map frame:
\begin{align*}
\begin{bmatrix}
x \\
y
\end{bmatrix}_{m,t} = 
\begin{bmatrix}
x \\
y
\end{bmatrix}_{m,t-1} +
\begin{bmatrix}
\Delta x \\
\Delta y
\end{bmatrix}_{m}
\end{align*}
and $\theta_{m,t} = \theta_{m,t-1} + \Delta \theta$
\end{enumerate}

\end{document}