\documentclass[11pt]{article}
\setlength{\oddsidemargin}{0.1 in}
\setlength{\evensidemargin}{-0.1 in}
\setlength{\topmargin}{-0.75 in}
\setlength{\textwidth}{6.5 in}
\setlength{\textheight}{9 in}
\setlength{\headsep}{0.75 in}
\setlength{\parindent}{0 in}
\setlength{\parskip}{0.2 in}

\usepackage{amsmath,amsfonts,graphicx,float, amssymb}
\usepackage{enumitem}
\usepackage[english]{babel}
\usepackage[square,numbers]{natbib}
\bibliographystyle{plainnat}
\usepackage{tikz}

\begin{document}
\section{Notation}
We make use of the following notation in this report.
\begin{itemize}
\item $r,p,o,m$ in subscript: Robot, particle, odometry and map frames, respectively
\item $R_{r,o}$: Rotation matrix expressing the orientation of the robot frame $r$ in the odometry frame $o$.
\item $R_{p,m}$: Rotation matrix expressing the orientation of the particle frame $p$ in the map frame $m$.
\end{itemize}

\section{Approach}
The following sections outline our approach in implementing the particle filter.

\subsection{Initialization}
We initialize the particles with uniformly drawn random positions and orientation in the unoccupied grid cells of the map, i.e. cells that are known to be empty with probability $1$.

\subsection{Motion Model}
\begin{enumerate}
\item Given subsequent odometry readings $\left(x,y,\theta\right)_{1}$ and $\left(x,y,\theta\right)_{2}$ at $t_{1}, t_{2}$, we first compute the translation in the odometry frame:
\begin{align*}
\begin{bmatrix}
\Delta x \\
\Delta y
\end{bmatrix}_{o} = 
\begin{bmatrix}
x_{2} - x_{1} \\
y_{2} - y_{1}
\end{bmatrix}
\end{align*} 

\item We then rotate the displacements to the robot frame (computed using odometry values):
\begin{align*}
\begin{bmatrix}
\Delta x \\
\Delta y
\end{bmatrix}_{r} = 
R_{r}^{-1}
\begin{bmatrix}
\Delta x \\
\Delta y
\end{bmatrix}_{o} 
\end{align*}

\item We add some noise to get corresponding displacements in the particle frame.
\begin{align*}
\begin{bmatrix}
\Delta x \\
\Delta y
\end{bmatrix}_{p} = 
\begin{bmatrix}
\Delta x \\
\Delta y
\end{bmatrix}_{r} +
\begin{bmatrix}
\mathcal{N}\left( 0, \sigma_{motion} \right) \\
\mathcal{N}\left( 0, \sigma_{motion} \right)
\end{bmatrix}_{}
\end{align*}
Similarly, the difference in $\theta$ is computed according to:
\begin{align*}
\Delta \theta = \theta_{2} - \theta_{1} + \mathcal{N}\left( 0, \sigma_{angle} \right)
\end{align*}

\item We rotate the displacement from the particle frame to the map frame:
\begin{align*}
\begin{bmatrix}
\Delta x \\
\Delta y
\end{bmatrix}_{m} = 
R_{p} * \begin{bmatrix}
\Delta x \\
\Delta y
\end{bmatrix}_{p}
\end{align*}

\item Finally, we update the particle's pose in the map frame:
\begin{align*}
\begin{bmatrix}
x \\
y
\end{bmatrix}_{m,t} = 
\begin{bmatrix}
x \\
y
\end{bmatrix}_{m,t-1} +
\begin{bmatrix}
\Delta x \\
\Delta y
\end{bmatrix}_{m}
\end{align*}
and $\theta_{m,t} = \theta_{m,t-1} + \Delta \theta$
\end{enumerate}

\subsection{Sensor Model}

\subsection{Resampling}

\section{Parameters and Tuning}
This section describes the parameters we've used in our particle filter, their values, and how we tuned them.

\subsection{Initialization}
\begin{itemize}
\item $\text{num\_particles}$: We use 80000 particles in our implementation. This seemed to be a suitable number of particles that covered the range of possible initial positions and orientations fairly well.
\end{itemize}

\subsection{Motion Model}
\begin{itemize}
\item $\sigma_{motion}, \sigma_{theta}$: Standard deviation of the gaussian used to add noise to motion model updates. We used a value of 6 for $\sigma_{motion}$ and $0.004$ for $\sigma_{theta}$ after tuning to ensure that our particles covered the span of possible motion updates at each time step, given that the odometry data was noisy.
\end{itemize}

\subsection{Sensor Model}
\begin{itemize}
\item $\sigma_{laser}$: The standard deviation of the gaussian used to generate one part of $P\left( z|x \right)$. A value of $200$ was found to be suitable for dealing with sensor noise after testing with particles initialized in different regions of the map.
\item $\text{MIN\_PROBABILITY\_VALUE}$: The minimum probability assigned to $P\left( z|x \right)$. We used a value of $0.1$ to avoid extremely small weights for laser readings that did not correspond to the map.
\item $\text{LASER\_HOP}$: This defines the sampling density of laser readings. We examine laser readings at intervals of $10$ degrees to avoid large processing times.
\item $\text{AT\_WORLDS\_END}$: The range beyond which we consider an end-of-range situation for a particular laser reading. We found a value of $700$ to be suitable.
\item $\text{EOR\_PROB}$: The value of $P\left( z|x \right)$ assigned to laser readings that fall beyond $\text{AT\_WORLDS\_END}$. A value of $0.3$ lets us assign non-trivial weights to laser readings that are not erroneous, but just happened to fall on unoccupied regions of the map.
\item $\text{GAUSSIAN\_MULTIPLIER}$: The multiplier applied to the gaussian portion of the sensor model. A value of 700 allows us to achieve sufficiently high probability values for laser readings that correspond well with the map.
\item $\text{EXP\_MULTIPLIER}$: A multiplier of $0.2$ lets us downweight the exponential portion of the model used to model objects that are very close to the robot.
\end{itemize}

\subsection{Resampling}

\end{document}